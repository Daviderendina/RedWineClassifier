\documentclass{article}
\usepackage[utf8]{inputenc}

\title{RelazioneProgettoML}
\author{d.rendina2 }
\date{February 2021}

\begin{document}

\maketitle

\section{Esplorazione del dataset e PCA}
    
    \subsection{Esplorazione del dataset}
         
    % Il dataset utilizzato è <link>, che caratterizza ecc.. scelto perché ecc..
         
    % Come intendiamo utilizzarlo: nel senso, che tipo di predizione vogliamo fare?
    
    Vogliamo caratterizzare le istanze in base alle loro feature per poi classificarle in 2 classi, in base alle feature.
    \\
    % Descrizione dataset: quali feature presenta
         
    Al fine di effettuare le predizioni, è sono state effettuate alcune modifiche:
    \begin{itemize}
        
        \item Sono state rimosse le feature \textit{ID} e \textit{name}, in quanto relative al singolo progetto e non permettono quindi di effettuare una classificazione accurata
        
        \item Convertito il valore del campo \textit{goal} dalla valuta indicata nella feature \textit{currency} a USD 
        
        \item Rimosso il campo \textit{currency}, poiché è stato convertito tutto in USD
        
        \item Rimossi i campi \textit{deadline} e \textit{launched} e sostituiti con un'unica feature \textit{duration\_in\_days}, che indica appunto la durata totale della campagna di crowdfunding (calcolata come differenza tra deadline e launched). + teniamo anno/mese ? Direi che non serve. Questa aggiunta è stata fatta poiché le date di inizio e fine sono irrilevanti ai fini della predizione, ma è più significativo conoscenere la durata di una campagna per uniformare questi due campi
        item
        
        \item \textit{pleged} lo toglierei, poiché non avendo informazioni sui giorni che mancano alla fine della campagna, questo campo risulterebbe inutile. Inoltre l'obiettivo della nostra classificazione è stabilire se una campagna potrà avere successo o meno, ma questa feature risulterebbe inutile per questo obiettivo (basterebbe controllare se pleged > goal e ho fatto). L'unica utilità sarebbe appunto per sottoporre all'algoritmo istanze dicendo: ok mi mancano tot giorni e ho raccolto x, avrò successo? Ma come già spiegato questo sarebbe impossibile, poichè mancano diverse informazioni per questo (in quanti giorni è stata raggiunta, perchè è stato cancellato e nel caso quanti soldi aveva raccolto, ecc..) \textbf{NB: non è testo definitivo ovviamente, questo è proprio buttato lì per esporre la mia idea}
        
        \item \textit{state} La variabile rimane così com'è, ma sono stati effettuate alcune scelte basate sul suo valore:
            \begin{itemize}
            
                \item \textit{live} e \textit{undefined}: rimossi (per stesso motivo di pleged - forse non molto chiaro, e perchè se è undefined non ci serve a nulla)
                
                \item \textit{successful} e \textit{failed}: considerati
                
                \item \textit{suspended} e \textit{canceled}? Si potrebbero tenere e considerarli come progetti che comunque sono falliti (anche se non avendo la causa del fallimento, non sappiamo effettivamente cosa togliamo) oppure toglieri (ne abbiamo già molti di progetti falliti e non, inoltre il dataset risulterebbe più bilanciato)
                
            \end{itemize}
        
        \item \textit{backers} Si riferisce sempre al numero di backers ma lo scarterei per gli stessi motivi di \textit{pleged}.
        
        \item \textit{country} invece ritengo sia utile, al massimo verrà scartato dalla PCA 
        
        \item \textit{usd\_pleged} rimosso per stesso motivo di pleged, + utlime 3 colonne senza nome rimosse
        
    \end{itemize}
    
    \subsubsection{Feature utilizzate}
    
        Per la classificazione (che prima passa dalla PCA), saranno quindi usati:
        \begin{itemize}
            \item category
            \item main\_category
            \item duration\_in\_days
            \item goal
            \item country
        \end{itemize}
        
        La nostra variabile target sarà invece \textit{state}.
    
\end{document}
